\documentclass{RT_Enzyme-article}

\begin{document}

\title{基于XeCJK的中文作业模板文件}
\author{Nbody}
%\date{} % 若不需要自动插入日期,则去掉前面的注释;{ } 中也可以自定义日期格式
\maketitle

\section{模板介绍}
这个模板是UTF-8编码的,使用xeCJK宏包,中英文混排更美观,但编译速度稍慢。注意:该模板只能用xelatex编译。常用逻辑符号命令:
\begin{center}
\begin{tabular}{|c|c|c|}
\hline
名称      & 符号                   & 命令 \\
\hline
否定      & $\neg$                & \verb|\neg| \\
合取      & $\land$               & \verb|\land| \\
析取      & $\lor$                & \verb|\lor| \\
蕴含      & $\to$                 & \verb|\to| \\
推出      & $\Rightarrow$         & \verb|\Rightarrow| \\
实质蕴涵   & $\supset$             & \verb|\supset| \\
双蕴含     & $\leftrightarrow$     & \verb|\leftrightarrow| \\
等价      & $\equiv$              & \verb|\equiv| \\
存在量词   & $\exists$             & \verb|\exists| \\
全称量词   & $\forall$             & \verb|\forall| \\
必然      & $\Box$                & \verb|\Box| \\
可能      & $\Diamond$            & \verb|\Diamond| \\
可满足,真  & $\models$            & \verb|\models| \\
语义后承   & $\vDash$,$\nvDash$   & \verb|\vDash,\nvDash| \\
句法后承   & $\vdash$,$\nvdash$   & \verb|\vdash,\nvdash| \\
属于      & $\in$,$\notin$       & \verb|\in,\notin| \\
真包含于   & $\subset$             & \verb|\subset| \\
包含于     & $\subset$            & \verb|\subset| \\
不等于     & $\neq$               & \verb|\neq| \\
小于等于   & $\leq$               & \verb|\leq| \\
大于等于   & $\geq$               & \verb|\geq| \\
下省略    & $\ldots$              & \verb|\ldots| \\
中省略    & $\cdots$              & \verb|\cdots| \\
对角省略  & $\ddots$              & \verb|\ddodts| \\
\hline 
\end{tabular}
\end{center}
更多符号对应的命令请参见:
\begin{itemize}
  \item \url{https://oeis.org/wiki/List_of_LaTeX_mathematical_symbols}或
  \item \url{https://www.ctan.org/tex-archive/info/symbols/comprehensive/}
\end{itemize}


\section{中文字体}

\subsection{字体切换}
默认字体为宋体。我们可以这样来改变中文字体:\heiti 从现在起是黑体。\kaiti 从现在起是楷体。\fangsong 从现在起是仿宋。\lishu 从现在起是隶书。\yuanti 从现在起是圆体。\songti 从现在起又是宋体。

\subsection{字体强调}
加粗字体自动变为黑体:\bf{粗体},加斜或强调字体自动变成楷体:\it{斜体},\emph{强调}。


\section{习题环境}

\begin{exs}
请证明勾股定理。
\end{exs}
\begin{proof}
这是证明。末尾后会自动添加方块以示结束。
\end{proof}

\begin{exs}
请计算 $1+2+\ldots +100$。
\end{exs}
\begin{proof}[解答]
这是解答。末尾后会自动添加方块以示结束。
\end{proof}

\end{document}