\NeedsTeXFormat{LaTeX2e}
\ProvidesClass{RT_Enzyme-article}[29-10-2022 v1.0 RT_Enzyme private class]
\typeout{This is LaTeX template for article. Version 1.0 (based on CTeX) 03-05-2021}

\LoadClass[UTF8,0.825318cm]{article}
\RequirePackage{amssymb,amsfonts,amsmath,amsthm,latexsym}

\RequirePackage{xeCJK}
\RequirePackage{listings}
\RequirePackage{type1cm}

\setCJKmainfont[BoldFont={SimHei},ItalicFont={KaiTi}]{SimSun}
\setCJKsansfont{SimHei}
\setCJKfamilyfont{zhsong}{SimSun}
\setCJKfamilyfont{zhhei}{SimHei}
\setCJKfamilyfont{zhkai}{KaiTi}
\setCJKfamilyfont{zhfs}{FangSong}
\setCJKfamilyfont{zhli}{LiSu}
\setCJKfamilyfont{zhyou}{YouYuan}
\setCJKfamilyfont{deyi}{Smiley Sans}


\newcommand*{\songti}{\CJKfamily{zhsong}} % 宋体
\newcommand*{\heiti}{\CJKfamily{zhhei}}   % 黑体
\newcommand*{\kaiti}{\CJKfamily{zhkai}}  % 楷体
\newcommand*{\fangsong}{\CJKfamily{zhfs}} % 仿宋
\newcommand*{\lishu}{\CJKfamily{zhli}}    % 隶书
\newcommand*{\yuanti}{\CJKfamily{zhyou}} % 圆体
\newcommand*{\deyi}{\CJKfamily{deyi}} % 圆体

\newcommand{\xiaosi}{\fontsize{12pt}{18pt}\selectfont}  %小四
%----------
% 版面设置
%----------
%首段缩进
\usepackage{indentfirst}
\setlength{\parindent}{2em}

%行距
% \renewcommand{\baselinestretch}{1.5} % 1.4倍行距

%页边距
\RequirePackage[a4paper]{geometry}
\geometry{verbose,
  tmargin=2cm,% 上边距
  bmargin=2cm,% 下边距
  lmargin=3cm,% 左边距
  rmargin=3cm % 右边距
}


%----------
% 其他宏包
%----------
%图形相关
\RequirePackage[x11names]{xcolor} % must before tikz, x11names defines RoyalBlue3
\RequirePackage{graphicx}
\RequirePackage{pstricks,pst-plot,pst-eps}
\RequirePackage{subfig}
\def\pgfsysdriver{pgfsys-dvipdfmx.def} % put before tikz
\RequirePackage{tikz}

%原文照排
\RequirePackage{verbatim}

%网址
\RequirePackage{url}

%----------
% 习题与解答环境
%----------
%习题环境
\theoremstyle{definition} 
\newtheorem{exs}{\bf{习题}}

%解答环境
\ifx\proof\undefined\
\newenvironment{proof}[1][\protect\proofname]{\par
\normalfont\topsep6\p@\@plus6\p@\relax
\trivlist
\itemindent\parindent
\item[\hskip\labelsep
\scshape
#1]\ignorespaces
}{%
\endtrivlist\@endpefalse
}
\fi

\renewcommand{\proofname}{\bf{解}}


% 用来设置附录中代码的样式
\RequirePackage{graphicx}
\RequirePackage{pythonhighlight}

% ------------------******-------------------
% 摘要
\renewcommand{\abstractname}{\Large{\heiti 摘要}}

% 引用跳转*
